\chapter{Conclusion}

This report presented the design, simulation, and analysis of a microstrip patch antenna enclosed within a flat radome structure. The study covered theoretical foundations of antenna radiation, the role of radomes in protecting antenna systems, and the interaction between the antenna and dielectric enclosure.

Patch antennas were chosen due to their low profile, ease of fabrication, and suitability for compact systems. A flat radome configuration was analyzed in detail, focusing on its impact on gain, polarization purity, and far-field radiation characteristics. Simulation results confirmed that the presence of the flat radome introduces measurable but manageable effects on antenna performance. Key parameters such as co- and cross-polarization behavior, beam shape, and gain patterns were evaluated with and without the radome to quantify this influence.

While the project was limited to flat radome geometries, the insights obtained here lay the groundwork for broader investigations into other radome types and materials. The methodology and results can directly inform the design of robust, high-performance antennas for radio astronomy and other applications operating in demanding environmental conditions.
