\chapter{Introduction}

The rapid growth of wireless communication, satellite systems, and high-frequency radio astronomy experiments has created an increasing demand for compact, efficient, and cost-effective antenna solutions. Microstrip patch antennas have become highly attractive candidates due to their low profile, ease of fabrication, and compatibility with modern printed circuit technologies \cite{werfelli2016patch,balanis}. These antennas are well-suited for integration in arrays and conformal platforms, supporting applications ranging from telecommunications to scientific observation.

However, microstrip patch antennas are sensitive to their surrounding environment. Environmental factors such as rain, wind, dust, or mechanical impacts can degrade antenna performance or even cause damage. To overcome these challenges, antennas are typically housed within protective structures known as radomes. A radome is a structural, weatherproof enclosure that shields the antenna while aiming to minimally affect its electromagnetic performance \cite{swra705,aemterms}. Ideally, a radome should be transparent to the antenna’s operating frequency, maintaining the integrity of the radiation pattern and minimizing losses.

In radio astronomy applications, where extremely high sensitivity is required, the impact of radomes becomes even more critical. Subtle distortions introduced by the radome can affect beam shapes, sidelobe levels, or polarization purity, thereby impacting the accuracy of scientific measurements \cite{mcculloch2023sband}. Moreover, as the operating frequency decreases toward radio wavelengths, the radome’s physical thickness increases to maintain structural integrity, which may result in additional signal attenuation or pattern distortion.

This project focuses on understanding the design principles of patch antennas and analyzing the effect of various radome configurations on antenna performance. Building on the foundation of electromagnetic radiation theory \cite{balanis}, the project will explore flat, spherical, and other radome geometries through simulation. The ultimate goal is to identify suitable radome materials and shapes that protect the antenna while preserving its desired electromagnetic characteristics.

\section*{Conclusion}

This introductory chapter outlined the motivation behind studying radome-antenna interactions in RF systems, particularly in the context of radio astronomy. It also introduced the challenges posed by environmental effects on high-frequency antennas and the need for careful radome design. With this context, the next chapter builds the theoretical foundation of antenna radiation, laying the groundwork for simulation and analysis.
